\documentclass[a4paper]{article}
\usepackage{hyperref}

\begin{document}
\section{Introduction}
\textbf{Company Tycoon} is team building exercise designed for teams of 4-5 players. The game simulates the tough decisions that must be made by a technology startup.

\section{Game Setup}
Running the game requires:

\begin{itemize}
    \item Gamemaster (GM)
    \item (Optional) Second person to walk around the room.\footnote{In play testing it is very useful to have someone answering questions so the GM can focus on running the game.} 
    \item Teams of 4-5 players
    \item The GM software (available at: \href{https://github.com/tawilkinson/company-tycoon}{https://github.com/tawilkinson/company-tycoon}), a cross-platform python program
    \item Monopoly money or another way of visualising currency
    \item Decision points card specific to your game
\end{itemize}

\subsection{Roles}

\begin{description}
    \item [CEO] Chief Executive Officer - Has deciding vote when team cannot agree.
    \item [MD] Managing Director - Must keep track of the time and keep team on track.
    \item [CTO] Chief Technical Officer - Can pick one \textbf{Decision Point} to focus their attention on. Reduces R\&D time for that project by 1 month.
    \item [CFO] Chief Financial Officer - Must bring all decisions to \textbf{GM} with the money to pay for them. Keeps track of team finances.
    \item [Marketing] (Optional for when 5 team members) For a cost\footnote{Set a reasonable price based on the costs in your game} can apply a multiplier to one system upgrade.
\end{description}

\section{Mechanics}

\subsection{Time}

This game runs well in 1 to 1.5 hours including a brief explanation. You can set the game time in the GM software. \textit{Default:} 1 hour 15 minutes.

%\subsection{Time}

\section{Decision Points}
\textbf{Decision Points} are a key part of the game that reflect the tough decisions a real company must make. Once a team has made a decision has been selected by the team the CFO must come to the GM to pay for the development.
\subsection{Required}
Before teams can begin selling they must have researched the minimum components to make their device. All of the required decisions must be made before a team can generate revenue.\footnote{Remind them the clock is ticking!} A team may revisit a required choice later and pay any additional R\&D time and cost to upgrade that component.
In general:
\begin{itemize}
    \item Required items should reflect components of the minimum viable product.
    \item Think carefully about the development time. Technology that is off-the-shelf should have little to no research time but likely sells for less.
    \item There's no reason why a required component can't reduce the overall sales price but this should be balanced by a larger modifier to sales chance as cheaper products, in general, sell better.
\end{itemize}
\subsection{Upgrades}
Upgrades increase the selling cost of the company's device. Functionally they imply a modifier to the selling price that the GM sets in the config ahead of the game. Players do not need to know the exact modifier but should have a hint as to whether an upgrade increases sales chance and it should list the change in list price of the product. While upgrades can be researched before required items.

In general:
\begin{itemize}
    \item Upgrades that significantly increase sales chance and sales price should cost more money and take longer to develop.
    \item Some of the upgrades should only increase price or sales chance but not both.
    \item There is no problem in having multiple options that have a small effect on either sales chance or sales price. Researching lots of quick/cheap things to increase your odds is a solid strategy.
\end{itemize}
% \section{Glossary}*
\end{document}
